\documentclass{article}
 
\usepackage{polski}
\usepackage[utf8]{inputenc}
\usepackage{amsmath}
\usepackage{bm}
\usepackage[linguistics]{forest}
 
\begin{document}
\title{Optymalne mnożenie ciągu macierzy}

\maketitle

\section{Treść}

Mamy ciąg $ N+1 $ liczb $ (a_{0}, a_{1}, \dots, a_{N - 1}, a_{N}) $. \\
Mamy również ciąg $ N $ macierzy $ (M_{0}, M_{1}, \dots, M_{N - 2}, M_{N-1}) $ \\
Macierz $ M_{i} (a_{i} \times a_{i+1}) $ ma $a_{i} $ wierszy i $ a_{i+1} $ kolumn. \\ \\
Należy znaleźć takie nawiasowanie ciągu, aby liczba wykonanych mnożeń (liczb) była minimalna. \\ Zakładamy, że przemnożenie macierzy A $( n \times m $) oraz B $ (m \times r) $ polega na wykonaniu $ m * n * r $ mnożeń. (Standardowy algorytm mnożenia macierzy.)

Należy napisać program, który:
\begin{enumerate}
\item Obliczy minimalną liczbę mnożeń.
\item Wypisze optymalne nawiasowanie. np. $ (((0) * (1)) * (2)) = (0 * 1) * 2 $

% \item (TODO???) Wypisze przehashowany napis optymalnego nawiasowania.
% \item (TODO) n zapytań dla przehashowanych ciagow
\end{enumerate}

\section{Przykładowe mnożenie}
\begin{forest}
  [*
    [*
     [0]
     [1]
    ]
    [*
      [2]
      [*
        [*
          [3]
          [*
            [4]
            [*
              [5]
              [6]
            ]
          ]
        ]
        [7]
      ]
    ]
  ]
\end{forest}
$ (0*1)*(2*((3*(4*(5*6)))*7)) $

\section{Przykładowe ciągi}
\subsection{ n = 1}
Nie ma czego mnożyć.

\subsection{ n = 2}
\begin{gather*}
n = 2 \\
a = (3,4, 5)
\end{gather*}
\subsubsection{Nawiasowanie $ (0 * 1) $}
Liczba mnożeń wynosi: $ 3 * 4 * 5 = 60 $ 


\subsection{n = 3}
\begin{gather*}
n = 3 \\
a = (10^6,10^6, 10^6, 1)
\end{gather*}
\subsubsection{Nawiasowanie $ (0 * 1) * 2 $}
Liczba mnożeń wynosi: $ 10^6 * 10^6 * 10^6 + 10^6 * 10^6 * 1 = 10^{18} + 10^{12} $ 

\subsubsection{Nawiasowanie $ 0 * (1 * 2) $}
Liczba mnożeń wynosi: $ 10^6 * 10^6 * 1 + 10^6 * 10^6 * 1 = 2 * 10^{12} $

\subsection{n = 5}
\begin{gather*}
n = 5 \\
a = (30,1000,1000,1000,4,3)
\end{gather*}
\subsubsection{Nawiasowanie $ (((0 * 1) * 2) * 3) * 4 $}
Liczba mnożeń wynosi: $ 30 * 1000 * 1000 + 30 * 1000 * 1000 + 30 * 1000 * 4 + 30 * 4 * 3  = 30.000.000 + 30.000.000 + 120.000 + 360 = 60.120.360 $

\subsubsection{Nawiasowanie $ 0 * (1 * (2 * (3 * 4))) $}
Liczba mnożeń wynosi: $ 1000 * 4 * 3 + 1000 * 1000 * 3 + 1000 * 1000 * 3 + 30 * 1000 * 3  =
12.000 + 3.000.000 + 3.000.000 + 90.000 = 6.102.000 $

\subsubsection{Nawiasowanie $  0 * (((1 * 2) * 3) * 4) $}
Liczba mnożeń wynosi $ 1000 * 1000 * 1000 + 1000 * 1000 * 4 + 1000 * 4 * 3 + 30 * 1000 * 3 =
1.000.000.000 + 4.000.000 + 12.000 + 90.000 = 1.004.102.000 $

\section{Rozwiązanie cz. I $ O(n^3) $ }
Zastosujemy programowanie dynamiczne. \\
Niech $ dp(i, j) $ oznacza minimalną liczbę mnożeń dla podciągu macierzy $ (M_{i}, M_{i+1}, \dots , M_{j-1}, M_{j}). $
\begin{enumerate}
\item $ dp(i, j) $ jest nie określone, gdy $ j < i $.
\item Nietrudno zauważyć, że $ dp(i, i) = 0 $. \\Mając tylko jedną macierz nie musimy jej mnożyć. 
\item $ dp(i, j) = min_{k \in \{ i, i + 1,  \dots, j-1 \} } dp(i, k) + dp(k + 1, j) + a_i * a_{k +1} * a_{j + 1} $ \\
Dzielimy ciąg na dwa ciągi, które osobno wymnarzamy. Następnie ze wszystkich podziałów wybieramy najlepszy. \\
\begin{gather*}
(M_{i}, \dots, M_{k}) \rightarrow A(a_i \times a_{k+1}) \\
(M_{k+1}, \dots, M_{j}) \rightarrow B(a_{k+1} \times a_{j+1}) \\
(M_{i}, \dots, M_{j}) \rightarrow A*B (a_i \times a_{j + 1})
\end{gather*}

\item Wartości obliczamy w kolejności rosnących różnic między indeksami. (tj. $ (j - i) $)
\item Wynikiem jest $ dp(0, n-1) $

\end{enumerate}

\section{Rozwiązanie cz. II}
Jak odtworzyć nawiasowanie?
\subsection{argmin}
Zwraca wartość ze zbioru X, taką, że wartość funkcji jest najmniejsza dla elementu $ k \in X $.

\begin{displaymath}
	k := argmin_{x \in X} f(x) \iff f(k) = min_{x \in X}f(x)
\end{displaymath}

\subsection{Rozwiązanie $ O(n^3) $}
Wystarczy dla każdego podciągu macierzy $ (M_{i}, \dots M_{j}) $ wyznaczyć punkt podziału k.
\begin{displaymath}
k := argmin_{k \in \{ i, i + 1,  \dots, j-1 \} } dp(i, k) + dp(k + 1, j) + a_i * a_{k +1} * a_{j + 1}
\end{displaymath}
Wówczas poszukiwane nawiasowanie wygląda następująco.
\begin{enumerate}
\item nawias(i, i) = i
\item nawias(i, j) = (nawias$(i, k)$)$*$(nawias$(k+1, j)$)
\item Wynikiem jest nawias$(0, n-1)$.

\end{enumerate}

\subsection{Znajdowanie argmin}
Można to zrobić na dwa sposoby.
\begin{enumerate}
\item Mając obliczoną tablicę dp obliczamy ją dopiero wtedy go ją potrzebujemy po prostu iterując.
\item Obliczając tablice $dp$, możemy od razu zapamiętać w osobnej tablicy $ dpk$ odpowiednie wartości. (Wówczas cz. II działa w $ O(n) $)

\end{enumerate}

\subsection{Dowód złożoności}
\begin{gather}
	\sum_{i = 0}^n i = 0 + 1 + \dots + n = \frac{n(n+1)}{2} \\
	\sum_{i = 0}^n i^2 = \frac{n(n+1)(2n+1)}{6} \\
	\sum_{i = 0}^n i^3 = (\frac{n(n+1)}{2})^2 \\
	\sum_{i = 0}^n i^k = \dots =  O(n^{k+1})
\end{gather}
\begin{equation}
\begin{split}
\sum_{0 <= i <= j <=n} j - i & = \sum_{j = 0}^{n} \sum_{i=0}^{j} (j - i) \quad\text{(robimy sumę po dwóch indeksach)}  \\
&= \sum_{j = 0}^{n}(\sum_{i=0}^j j - \sum_{i = 0}^j i) \quad\text{(rozbijamy na dwie sumy)}\\
&= \sum_{j = 0}^n j(j+1) - \frac{j(j+1)}{2} \quad\text{(obliczamy sumy osobno)} \\
&= \sum_{j = 0}^n \frac{j(j+1)}{2} = \sum_{j = 0}^n \frac{j^2}{2} + \frac{j}{2} \quad\text{(wyciągamy j(j+1) przed nawias i upraszczamy)}\\
&= \frac{\sum_{j = 0}^n j^2}{2} + \frac{\sum_{j = 0}^n j}{2} \quad\text{(sprowadzamy do wzoru na sumy)} \\
&= \frac{n(n+1)(2n+1)}{12} + \frac{n(n+1)}{4} \quad\text{(podstawiamy)} \\
&= O(n^3 + n^2) = O(n^3)
\end{split}
\end{equation}



\end{document}

