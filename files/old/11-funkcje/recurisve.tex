\documentclass{article}
 
\usepackage{polski}
\usepackage[utf8]{inputenc}
\usepackage{amsmath}
 
\begin{document}
\title{Ściąga o równaniach rekurencyjnych}

\maketitle

\section{Grupa}
\begin{displaymath}
	(G, +, 0)
\end{displaymath}
G - zbiór, + operacja dwuagrumentowa, 0 - element neutralny względem dodawania
\begin{gather}
\forall_{a, b \in G} a + b \in G \\
\forall_{a, b, c \in G} (a + b) + c = a + (b + c) \\
\exists_{e \in G} \forall_{a \in G} a + e = e + a = a \\
\forall_{a \in G} \exists_{b \in G} a + b = b + a = e
\end{gather}

\section{Pierścień}
\begin{displaymath}
	(G, +, *, 0, 1)
\end{displaymath}
G - zbiór, + i * - operacje dwuargumentowe, 0 - element neutralny względem dodawania, 1 - element neutralny względem mnożenia
\begin{gather}
	(G, +, 0)	
\end{gather}

\section{Szybkie potęgowanie}
\begin{gather}
	n := \sum_{i=0}^{\lfloor log(n) \rfloor} a_i * 2^i, a_i \in \{0, 1\} \\
	k^n = \prod_{i=0}^{\lfloor log(n) \rfloor} k^{a_i * 2^i} = k^{a_0} * k^{2a_1} * k^{4a_2} * \dots
\end{gather}

\section{Równania rekurencyjne (RR)}
\subsection {RR drugiego rzędu}
$((p_0, p_1), (a_0, a_1))$,  $ p_i $ - parametry równania, $a_i $ - wartości początkowe
\begin{gather}
a_0 = ?, a_1 = ? \\
\forall_{n > 1}a_n = p_1 * a_{n-1} + p_0 * a_{n-2}
\end{gather}
\subsubsection{Fibonacci}
\begin{gather*}
a_0 = a_1 = p_0 = p_1 = 1 \\
a_n = a_{n-1} + a_{n-2}
\end{gather*}
1, 1, 2, 3, 5, 8, 13 \dots

\subsubsection{Lucas}
\begin{gather*}
a_0 = 2, a_1 = 1 \\
p_0 = p_1 = 1 \\
a_n = a_{n-1} + a_{n-2}
\end{gather*}
2, 1, 3, 4, 7, 11, 18 \dots

\subsubsection{Wersja macierzowa}
\begin{gather*}
S = \begin{bmatrix}
a_1 & a_0 \\
a_0 & 0
\end{bmatrix} \\
I = \begin{bmatrix}
p_1 & p_0 \\
p_0 & 0
\end{bmatrix} \\
SI^n = \begin{bmatrix}
a_{n+1} & a_{n} \\
a_{n} & a_{n-1}
\end{bmatrix}
\end{gather*}

\subsection {RR k-tego rzędu}
$((p_0, p_1, \dots, p_{k-1}), (a_0, a_1, \dots, a_{k-1})) $
\begin{gather*}
\forall_{n > k - 1} a_n = \sum_{i=1}^{k} a_{n - i} p_{k - i} = a_{n-1} p_{k - 1} + a_{n-2} p_{k-2} + \dots + a_{n-k} p_{0}
\end{gather*}

\subsubsection{Wersja macierzowa}
\begin{gather*}
S = \begin{bmatrix}
a_{k-1} & a_{k-2} & a_{k-3} & \dots & a_{1} & a_{0} \\
a_{k-2} & a_{k-3} & a_{k-4} & \dots & 0 & 0 \\
\vdots & \vdots & \vdots & \ddots & \vdots & \vdots \\
a_{1} & a_{0} & 0 & \dots & 0 & 0 \\
a_{0} & 0 & 0 & \dots & 0 & 0
\end{bmatrix} \\
I = \begin{bmatrix}
p_{k-1} & p_{k-2} & p_{k-3} & \dots & p_{1} & p_{0} \\
p_{k-2} & p_{k-3} & p_{k-4} & \dots & 0 & 0 \\
\vdots & \vdots & \vdots & \ddots & \vdots & \vdots \\
p_{1} & p_{0} & 0 & \dots & 0 & 0 \\
p_{0} & 0 & 0 & \dots & 0 & 0
\end{bmatrix} \\
SI^n = \begin{bmatrix}
a_{n + k-1} & a_{n + k-2} & a_{n + k-3} & \dots & a_{n + 1} & a_{n} \\
a_{n + k-2} & a_{n+k-3} & a_{n+k-4} & \dots & a_{n} & 0 \\
\vdots & \vdots & \vdots & \ddots & \vdots & \vdots \\
a_{n + 1} & a_{n} & 0 & \dots & 0 & 0 \\
a_{n} & 0 & 0 & \dots & 0 & 0
\end{bmatrix}
\end{gather*}
Coś nie działają te równania



\end{document}

