\documentclass{article}
 
 
\usepackage{polski}
\usepackage[utf8]{inputenc}
\usepackage{amsmath}
\usepackage{amssymb}
\usepackage{listings}
\usepackage{array}
\usepackage{verbatim}
\lstset{language=C++}
 
\begin{document}
\title{Matematyczne szlaczki}

\maketitle

\section{Matematyczne szlaczki}
\begin{tabular}{ | l | l | m{3cm} | }
\hline
Napis matematyczny & Kod w C++ & Uwagi \\
\hline
$[a_{i}]_{i = 0, \dots, N-1 }$ &
\begin{lstlisting}
#define N 1003
int a[N];
\end{lstlisting}
& Tablica zawierająca N elementów. (Ciąg / Wektor N-elementowy) \\
& \begin{lstlisting}
const int N = 1003;
int a[N];
\end{lstlisting}

& Zmienna N deklarowana jako stała, a nie jako makro.\\
\hline

$[a_{i, j}]_{\substack{ i = 0, 1, 2, \dots, N-1 \\ j = 0, 1, 2 \dots, M-1}} $
& 
\begin{lstlisting}
const int N = 1003, M = 1004;
int a[N][M];
\end{lstlisting}
& Macierz $ N\times M $ \\
\hline

$ \forall_{i \in \{1, 2, \dots, n\}} $ & for(int i = 1; $ i <= n $; i++) & Pętla for \\
\hline

\begin{comment}
$ min_{i \in \{0, \dots, N-1\}} a_{i} $ & 
\begin{lstlisting}
#include<algorithm> // min
#include<climits> // INT_MAX
int m = INT_MAX;
for(int i = 0; i < N; i++)
	c = min(a[i], c);
\end{lstlisting}
& Jeśli a jest tablicą typu int o N elementach. \\
\hline
$ argmin_{i \in \{0, \dots, N-1\}} a_i $ &
\begin{lstlisting}
#include<algorithm> // min
#include<climits> // INT_MAX
int k = -1;
int m = INT_MAX;
for(int i = 0; i < N; i++)
	if(a[i] < m) {
		k = i;
		m = a[i];
	}
\end{lstlisting} & Pod zmienną k zapamiętujemy indeks dla którego wartość w tablicy jest najmniejsza. \\
\hline
\end{comment}
$
\sum_{i \in \{m, m+1, \dots, M \} } a_i
$
&
\begin{lstlisting}
int c = 0;
for(int i = m; i <= M; i++)
	c += a[i];
\end{lstlisting}

& \\
$\sum_{ i = m }^{M} a_i $ & & \\
\hline
$f(x, y) = x + y$ & 
\begin{lstlisting}
int f(int x, int y) {
	return x + y;
} 
\end{lstlisting}
& Funkcja biorąca dwie liczby jako parametr i zwracająca jedną liczbę (ich sumę). \\
\hline
$
f(x) = 
\begin{cases} 
3x + 1, & \text{gdy x nieparzyste} \\
\frac{x}{2} & \text{w przeciwnym wypadku}
\end{cases}
$
&
\begin{lstlisting}
int x = 17, y;
if(x % 2 == 1) {
	y = 3*x + 1;
}
else {
	y = x / 2;
}
\end{lstlisting}
& Instrukcje warunkowe \\
&
%\begin{lstlisting}
%y = (x % 2 == 1 ? 3 * x + 1 : x / 2);
%\end{lstlisting}
& \\
\hline
\end{tabular}

\section{Zadania}
\subsection{Rozwiązanie polega na napisaniu kodu}
We wszystkich poniższych zadaniach należy wyliczyć tablicę b na podstawie tablicy a.
\subsubsection{Ciąg Fibonacciego $O(n)$}
Tablica a jest dwuelementowa. n = 2018.
\begin{gather*}
a_0 = 0 \\
a_1 = 1 \\
\forall_{i \in \{ 0, 1 \} } b_i := a_i \\
\forall_{i \in \{2, 3, \dots, n\} } b_i := b_{i-1} + b_{i-2}
\end{gather*}
Wskazówka.
\begin{displaymath}
\forall_{i \in \{0, 1, \dots, n\} } b_{i} = 
\begin{cases} 
a_i, & \text{gdy $i < 2$} \\
b_{i-1} + b_{i -2} & \text{w przeciwnym wypadku}
\end{cases}
\end{displaymath}

\subsubsection{Suma prefiksowa $O(n)$}
\begin{displaymath}
	\forall_{k \in \{ 0, 1, \dots, n-1 \} } b_{k} := \sum_{i = 0}^{k} a_{i} 
\end{displaymath}
Wskazówka.
\begin{displaymath}
\forall_{i > 0} b_i = a_i + b_{i - 1}
\end{displaymath}

\subsubsection{* Tablica 3d $ O(n^3)$}
\begin{displaymath}
	\forall_{\substack{ 0 \leqslant x \\ x \leqslant y \\ y \leqslant z < n} }
	 b_{x, y, z} := x + y - z
\end{displaymath}
Wskazówka.
\begin{displaymath}
x \leqslant y, y \leqslant z, z < n \implies x \leqslant y \leqslant z < n \implies x < n, y < n
\end{displaymath}
Tam gdzie tablica b nie jest określona, wartość powinna być równa 0. \\
(Kiedy tablice w C++ mają wartość domyślną równą 0?)

\subsubsection{** Suma prefiksowa 2D $O(n^2)$}
Tablica a jest dwuwymiarowa $(n \times n)$
\begin{displaymath}
\forall_{\substack{ 0 \leqslant i < n \\ 0 \leqslant j < n}} b_{i, j} := \sum_{x = 0}^{i} \sum_{y = 0}^{j} a_{x, y}
\end{displaymath}
Rozwiązanie wprost ma złożoność $O(n^4)$. 
Można to zrobić szybiciej poprzez podzielenie zadania na mniejsze podzadania.
\begin{gather*}
\forall_{\substack{ 0 \leqslant i < n \\ 0 \leqslant j < n}} c_{i, j} := \sum_{y = 0}^{j} a_{i, y} = c_{i, j-1} + a_{i, j} \text{ (Liczymy sumę prefiksową wzdłuż kolumn. )} \\
\forall_{\substack{ 0 \leqslant i < n \\ 0 \leqslant j < n}} b_{i, j} := \sum_{x = 0}^{i} c_{x, j} \text{ (Liczymy sumę prefiksową wzdłuż wierszy dla wyników częściowych.) }
\end{gather*}

\subsubsection{*** Suma w prostokącie $O(1)$}
Mając policzoną tablicę b z zadania powyżej należy obliczyć wyrażenie.
\begin{displaymath}
f(x_1, y_1, x_2, y_2) := \sum_{ i = x_1 } ^ { x_2 } \sum_{ j = y_1 }^ {y_2} a_{i, j}
\end{displaymath}
Rozwiązanie. 
\begin{displaymath}
f(x_1, y_1, x_2, y_2) := b_{x_2, y_2} - b_{x_1 - 1, y_2} - b_{x_2, y_1 - 1} + b_{x_1 - 1, y_1 - 1}
\end{displaymath}

% $ \forall_{x \in X} $ & for(auto \&x : X) & Jeśli chcemy nadpisać x. \\
% & for(auto x : X) & Jeśli potrzebujemy tylko wartości x. \\
% \hline

\end{document}

