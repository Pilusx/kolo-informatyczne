\documentclass{article}
 
 
\usepackage{polski}
\usepackage[utf8]{inputenc}
\usepackage{amsmath}
\usepackage{amssymb}
\usepackage{listings}
\usepackage{array}
\usepackage{verbatim}
\usepackage{ntheorem}
\lstset{language=C++}

\theoremstyle{break}
\newtheorem{theorem}{Twierdzenie}
\newtheorem{corollary}{Wniosek}[theorem]
\newtheorem{lemma}{Lemat}
\newtheorem{proof}{Dowód}

\usepackage{bbm}
\usepackage{hyperref}
\hypersetup{
    colorlinks=true,
    linkcolor=blue,
    filecolor=magenta,      
    urlcolor=cyan,
}

 
\begin{document}
\title{Kontest 2}

\maketitle

\section{Zasady}
\begin{enumerate}
\item Można korzystać z dowolnych materiałów.
\item Kod ma być napisany samodzielnie tzn. 'kopiuj-wklej' jest niedozwolone, ale 'przepisz' już jest.
\end{enumerate}
\section{Wskazówki}
\begin{enumerate}
\item Rozwiązania nie powinny zająć więcej niż 60*2 linii (nie wliczając debuga).
\item \href{https://www.oi.edu.pl/old/php/show.php?ac=p173000&module=show&file=oi18/przekierowanie}{Przekierowywanie wejścia, wyjścia z pliku.}
\end{enumerate}

\section{Zadania cz. II}
\subsection{Wielkie Twierdzenie Fermata}
\subsubsection{Twierdzenia}
\begin{lemma}
\begin{displaymath}
p \in \mathbb{P} \implies \forall_{(a \mod{p}) \not\in \{0, 1\}, b, c} (a^b \equiv a^c \pmod{p} \iff b \equiv c \pmod{(p-1)})
\end{displaymath}
\end{lemma}

\begin{theorem}[Wielkie Twierdzenie Fermata]
$n > 2 \implies \lnot \exists_{0 < a, b, c}a^n + b^n = c^n$
\end{theorem}
\subsubsection{Właściwe zadanie}
Niech $k = 40097, m = 1000, p = 10^{9} + 7 \in \mathbb{P}$. \\
Niech $a,b,c \in \{k, k+1, k+2, \dots k+m\}$ \\
Należy wypisać jakąkolwiek krotkę $(a,b,c)$  taką, że:
\begin{displaymath}
a^{pk} + b^{pm} \equiv c^{p^2} \pmod{p}
\end{displaymath}
lub 'NIE', jeśli taka krotka nie istnieje. \\

Stopnie rozwiązania zadania:
\begin{enumerate}
\item Pokazanie, że $\forall_{k} pk \equiv k \pmod{p-1}$, oraz że $p^2 \equiv 1 \pmod{p-1}$
\item Pokazanie, że wystarczy sprawdzać $a^k + b^m \equiv c \pmod{p}$
\item $O(m^3 p)$
\item $O(m^3 log(p))$% $Wskazówka: pow.$
\item $O(m^2 log(m) + m log(p))$ % Wskazówka: $<algorithm>$ sort, find / $<set>$. 
\item ** Pokazanie, że gdyby zamiast $c^{p^2}$, wziąć $c^{p^2 - 1}$ i założyć, że:
\begin{displaymath}
k \equiv m \pmod{p-1},
\end{displaymath} (tj. $a^m + b^m \equiv 1 \pmod{p}$) to istnieje algorytm sprawdzający istnienie krotki w $O(mlog(m))$.
\end{enumerate}
Odpowiedzi: $([a,a^k \pmod{p}],[b,b^m \pmod{p}],[c,c])$ \\
([40723,675698371],[40482,675739095],[40724,40724]) \\
([40931,18032244],[41018,18072866],[40622,40622]) \\

\end{document}
