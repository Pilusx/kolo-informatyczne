\documentclass[paper=a4, fontsize=11pt]{article}
\usepackage[polish]{babel}
\usepackage{polski}
\usepackage[utf8]{inputenc}
\usepackage{amsmath,amsfonts,amsthm, hyperref}
\setlength\parindent{0pt} 

\title{	
\huge Programowanie dynamiczne na grafach
}

\date{\normalsize}

\begin{document}

\maketitle

\newtheorem{mydef}{Definicja}
\newtheorem{theorem}{Twierdzenie}
\newtheorem{lemma}{Lemat}

\section{Definicje}
\begin{mydef}[Rodzina indeksowana]
\begin{gather*}
\mathcal{A} = \{ A_i : A_i \subseteq X, i \in \mathbb{N} \} \\
\end{gather*}
Będziemy używać jedynie skończonych rodzin.
\end{mydef}
\begin{mydef}[Suma uogoólniona] 
\begin{gather*}
\bigcup \mathcal{A} = \bigcup_i A_i = \{ x \in X : \exists_i x \in A_i \}
\end{gather*}
\end{mydef}

\begin{theorem}[Wzór jawny dla ciągu Fibonacciego]
\begin{gather*}
F_0 = 0, F_1 = 1, F_n = F_{n-1} + F_{n-2} \\
F_n = \frac{1}{\sqrt{5}}(\frac{1 + \sqrt{5}}{2})^n - \frac{1}{\sqrt{5}}(\frac{1 - \sqrt{5}}{2})^n
\end{gather*}
\end{theorem}
\begin{proof}
n = 0, n = 1 sprawdzamy ręcznie. \\
Pokazujemy, że $x^2 = x + 1$, gdy $x \in \{\frac{1 + \sqrt{5}}{2}, \frac{1 - \sqrt{5}}{2}\}$ \\
Niech $x_1 = \frac{1 + \sqrt{5}}{2}, x_2 = \frac{1 - \sqrt{5}}{2}, p = \frac{1}{\sqrt{5}}$

\begin{equation*}
\begin{split}
F_n &= F_{n-1} + F_{n-2} = px_1^{n-1} - px_2^{n-1} + px_1^{n-2} - px_2^{n-2} \\
&= px_1^{n-2}(1 + x_1) + px_2^{n-2}(1 + x_2) \\
&= px_1^n + px_2^n
\end{split}
\end{equation*}
\end{proof}

\begin{mydef} [Programowanie dynamiczne na acyklicznych, skończonych grafach skierowanych]
Tworzenie algorytmu przebiega w następujący sposób. \\

Najpierw definiujemy zbiór X, pewną własność C oraz zbiory poprzedników $b(x)$ wszystkich elementów zbioru X. Każdy element x ma własność C lub jej nie ma.\\

Definiujemy również zbiór poprzedników danego zbioru jako.
\begin{displaymath}
B(X) = \bigcup_{x \in X} b(x)
\end{displaymath}

Dzielimy graf na warstwy \textbf{(sortowaniem topologicznym)}, zaczynając od wierzchołków do których nie wchodzi żadna krawędź (warstwa $W_0$). Wierzchołek v należy do warstwy $W_k$ wtedy i tylko wtedy, gdy jego \textbf{najdłuższa} ścieżka do wierzchołka z warstwy $W_0$ ma długość k. \\

Zakładamy tutaj, że wszyscy poprzednicy należą do poprzednich warstw. (Graf skierowany acykliczny.)
\begin{displaymath}
\forall_k \forall_{w \in W_k} b(w) \subseteq \bigcup_{i < k} A_i
\end{displaymath}

Naszym celem jest pokazanie, że jeśli dla wszystkich wcześniejszych wierzchołków zachodzi własność C i potrafimy na tej podstawie udowodnić własność C dla kolejnych wierzchołków, to własność C zachodzi dla wszystkich wierzchołków.
Dowodzimy to w następujący sposób.


\begin{enumerate}
\item Podstawa indukcji
\begin{displaymath}
\forall_{x \in W_{0}} C(x)
\end{displaymath} 
\item Krok indukcyjny
\begin{displaymath}
(\forall_{w \in W_n} (\forall_{p \in b(w)} C(p))) \implies C(w)
\end{displaymath}
\end{enumerate}

Wówczas zachodzi:
\begin{displaymath}
(\forall_{n \in \mathbb{N}}( \forall_{w \in W_n} (\forall_{p \in b(w)} C(p)) \implies C(w))) \implies \forall_{x \in X} C(x)
\end{displaymath}

Uwagi!
\begin{itemize}
\item Graf nie musi być spójny.
\end{itemize}
\end{mydef}

\end{document}
