\documentclass[paper=a4, fontsize=11pt]{article}
\usepackage[polish]{babel}
\usepackage{polski}
\usepackage[utf8]{inputenc}
\usepackage{amsmath,amsfonts,amsthm, hyperref}
\usepackage{listings}
\setlength\parindent{0pt} 

\title{	
\huge Zbiory i iteratory
}

\date{\normalsize\today}

\begin{document}

\maketitle

\newtheorem{mydef}{Definicja}
\newtheorem{theorem}{Twierdzenie}
\newtheorem{lemma}{Lemat}

\section{Wypisywanka}
Szablon pliku początkowego znajduje się w pliku zbiory\_start.cpp. \\

We wszystkich poniższych zadaniach na wejściu dostajemy liczbę \textit{n} oraz ciąg \textit{n} liczb postaci.\\
$\begin{matrix}
n & a_1 & a_2 & \dots & a_n
\end{matrix}$ \\
Przykładowy ciąg z pliku test.in
\lstinputlisting{test.in}

Od tego momentu uznajemy, że jeśli w zbiorze istnieją dwa takie same elementy to wykonujemy na nich operację tylko raz. Rozpatrujemy zbiór różnych elementów.

Na początku zobacz przykładowe użycie \href{https://en.cppreference.com/w/cpp/iterator/advance}{iteratorów}. Nastepnie przejdź do rozwiązywania zadań. 

\begin{enumerate}
\item Korzystając z funkcji \href{http://www.cplusplus.com/reference/set/set/insert/}{insert}. wczytaj liczby do zbioru. 

\item Wypisz liczby w porządku rosnącym przy użyciu pętli \href{https://en.cppreference.com/w/cpp/language/range-based for loop}{for-range}. Nie używaj funkcji begin() oraz end().

\item Korzystając z funkcji \href{http://www.cplusplus.com/reference/set/set/begin/}{begin()} i end().
\begin{enumerate}

\item Wypisz najmniejszą liczbę.
\item Wypisz 4 najmniejsze liczby, używając pętli...
\item Wypisz największą liczbę. Przydatny może się okazać operator- -.
\item Wypisz liczby w porządku rosnącym korzystając z iteratorów.
\end{enumerate}

\item Wypisz liczby korzystając z odwrotnych iteratorów oraz funkcji
\href{http://www.cplusplus.com/reference/set/set/rbegin/}{rbegin()} oraz rend(), 
\href{http://www.cplusplus.com/reference/set/set/erase/}{erase()}.
\begin{enumerate}
\item Wypisz 4 największe liczby.
\item Wypisz czwartą największą liczbę. Przydatna może się okazać funkcja \href{https://en.cppreference.com/w/cpp/iterator/advance}{advance()} z $<algorithm>$
\item Usuń 4 największe liczby.
\end{enumerate}

\item Wypisz rozmiar zbioru. \href{http://www.cplusplus.com/reference/set/set/size/}{size()}
\item Sprawdż przynależność do zbioru liczb 3 oraz 8.
\begin{enumerate}
\item Funkcją \href{http://www.cplusplus.com/reference/set/set/count/}{count()}
\item Funkcją \href{http://www.cplusplus.com/reference/set/set/find/}{find()}. Zwróć uwagę na to co zwraca funkcja find(), gdy element nie należy do zbioru.

\end{enumerate}


\item *Przetestuj działanie funkcji \href{http://www.cplusplus.com/reference/set/set/lower_bound/}{lower\_bound()} oraz  \href{http://www.cplusplus.com/reference/set/set/upper_bound/}{upper\_bound()} dla liczb 3 i 4.

\item **Przejrzyj \href{http://www.cplusplus.com/reference/set/set/set/}{dokumentację}, a następnie zmodyfikuj operator porównywania elementów w zbiorze, tak aby elementy większe były na początku. Zrób to tak, aby wszystkie poprzednie funkcje nadal działały.

\end{enumerate}

\section{Złożoność}
Przeczytaj paragrafy o złożoności ('complexity') oraz ważności iteratorów ('iterator validity') wszystkich wymienionych powyżej funkcji.

\section{Sortowanie}
\begin{enumerate}
\item Stwórz prostą strukturę uczeń (string imie, string nazwisko).
\item Posortuj uczniow przy pomocy $<set>$ pisząc odpowiednią funkcję porównującą. 
\item Posortuj uczniów przy pomocy \href{https://www.google.com/search?q=vector+stl}{vectora} i  funkcji \href{http://www.cplusplus.com/reference/algorithm/sort/}{sort}.
\end{enumerate}
Przy sortowaniu najistotniejsza jest kolejność nazwisk. Jeśli nazwiska są równe to interesuje nas kolejność imion. \\
Poniżej podane są przykładowy nieposortowany i posortowany ciąg uczniów. \\
(A, Nowak) (C, Kowalski) (B, Kowalski) \\
$\rightarrow$  (B, Kowalski) (C, Kowalski) (A, Nowak)


\end{document}
