\documentclass[paper=a4, fontsize=11pt]{article}
\usepackage[polish]{babel}
\usepackage{polski}
\usepackage[utf8]{inputenc}
\usepackage{amsmath,amsfonts,amsthm, hyperref}
\setlength\parindent{0pt} 

\title{	
\huge I/O w C
}

\date{\normalsize\today}

\begin{document}

\maketitle

\newtheorem{mydef}{Definicja}
\newtheorem{theorem}{Twierdzenie}
\newtheorem{lemma}{Lemat}

\section{Definicje}
\begin{mydef}Ciąg sum częściowych ciągu n liczb $(a_0, \dots, a_{n-1})$ to ciąg $(b_0, \dots, b_{n-1})$ taki, że:
\begin{displaymath}
	\forall_t b_t = \sum_{i=0}^{t} a_i = a_0 + a_1 + \dots + a_t
\end{displaymath}	

\end{mydef}

\begin{mydef}Średnia arytmetyczna ciągu n liczb $(a_0, \dots, a_{n-1})$
\begin{displaymath}
\sum_{i=0}^{n-1} \frac{a_i}{n} = \frac{a_0 + a_1 + \dots + a_{n-2} + a_{n-1}}{n} = \frac{b_{n-1}}{n}
\end{displaymath}

\section{Zadanie}
Waszym dzisiejszym zadaniem jest napisanie programu, który:


\end{mydef}

\begin{enumerate}
\item wczyta ze standardowego wejścia (stdin) jedną literę c oraz jedną liczbę naturalną n ($0 \le n < 5 * 10^6$) oddzieloną napisem \textit{:))-}
\item sprawdzi asercją czy wczytana liczba należy do odpowiedniego przedziału
\item sprawdzi asercją czy znak c jest małą literą alfabetu lub liczbą. Jeśli jest to ma przestać działać.
\item wczyta z pliku \textit{liczby.in} ciąg n liczb naturalnych $(a_0, \dots, a_{n-1})$ poodzielanych znakiem \textit{\#}.

\item wypisze do pliku \textit{pref000000.out} ciąg sum prefiksowych $(p_0, \dots, p_{n-1})$, gdzie pod 000000 należy podstawić liczbę n z bieżącymi zerami
\item wypisze na standardowe wyjście (stdout) ich średnią z dokładnością do 2 miejsc po przecinku (zawsze mają być 2 liczby po przecinku)
\end{enumerate}

Uwaga. 
\begin{enumerate}
\item Należy pamiętać o otwieraniu i zamykaniu plików.
\item Litera i cyfra nie muszą być oddzielone spacją :)
\item Tak. Na końcu napisu jest \textit{-} (Minus)
\item Po ostatniej liczbie znak \textit{\#} może nie wystąpić, ale wyjątkowo występuje :)
\item $0 \le a_i < 10^8 $
\item $13 -> 000013$
\item W pliku \textit{wejscia.txt} są dostępne przykładowe przypadki testowe wejścia (stdin).
\end{enumerate}

\section{Pliki nagłówkowe}
Można używać jedynie języka C oraz poniższych funkcji zawartych w odpowiednich plikach nagłówkowych. 

\subsection{$<stdio.h>$}
\begin{enumerate}
\item fopen
\item fclose
\item printf, fprintf, snprintf
\item scanf, fscanf
\end{enumerate}

\subsection{$<assert.h>$}
\begin{enumerate}
\item assert
\end{enumerate}

\subsection{$<ctype.h>$}
\begin{enumerate}
\item islower
\item isdigit
\end{enumerate}

\end{document}
