\documentclass[paper=a4, fontsize=11pt]{article}
\usepackage[polish]{babel}
\usepackage{polski}
\usepackage[utf8]{inputenc}
\usepackage{amsmath,amsfonts,amsthm, hyperref}
\usepackage{listings}
\setlength\parindent{0pt} 

\usepackage{mathtools}
\DeclarePairedDelimiter\ceil{\lceil}{\rceil}
\DeclarePairedDelimiter\floor{\lfloor}{\rfloor}

\title{	
\huge Programowanie dynamiczne
}

\date{\normalsize\today}

\begin{document}

\maketitle

\newtheorem{mydef}{Definicja}
\newtheorem{theorem}{Twierdzenie}
\newtheorem{lemma}{Lemat}
\newtheorem{observation}{Obserwacja}


\section{FCANDY (Wcale nie gra w minima)}
\subsection{Wejście}
\begin{description}
\item $N$ $(1 \leq i \leq N \leq 100)$ - liczba rodzajów cukierków.
\item $k_i$ $(1 \leq k_i \leq 500)$ - liczba cukierków danego rodzaju
\item $c_i$ $(1 \leq c_i \leq 200)$ - liczba kalorii danego cukierka
\end{description}
Dodatkowo zdefiniujmy:
\begin{description}
\item $K := max_i k_i$
\item $C := max_i c_i$
\end{description}

\subsection{O co chodzi?}

Mamy dany multizbiór S, należy znaleźć podział na dwa rozłączne podzbiory $P, P-S$, taki aby zminimalizować wartość bezwzględną różnic sum elementów w podzbiorach.

\begin{align*}
&f(P) := sum(P) - sum(P - S) \\
&\hat{f}(S) := min_{P \subseteq S} |f(P)|
\end{align*}

\subsection{Bruteforce $O(2^{NK})$}
Sprawdzamy wszystkie możliwe kombinacje P i wypisujemy tą z najmniejszą wartością $\hat{f}(P)$. Algorytm będzie działał w czasie $O(2^{NK})$
\begin{observation} Różnica dla zerowej liczby cukierków wynosi 0.
\begin{observation} $jc - (n - j)c  = (2j - n)c$

\end{observation}

\begin{lstlisting}[language=c++]
int n;
vector<int> k;
vector<int> c;
int brute(int wynik, int i) {
	if(i > n) return abs(wynik);
	int res = INT_MAX;	
	for(j = 1; j <= k[i], j++) {
		int czesc = (2 * j - k[i]) * c[i];
		res = min(res, brute(wynik + czesc, i+1);
	}
	return res;
}
res = brute(1);

\end{lstlisting}

\subsection{Programowanie dynamiczne $O(CK^2N^2)$}
\begin{observation} Liczba możliwych różnic jest ograniczona.
\begin{gather*}
	M_1 := CKN \leq 10^7 \\
	\forall_{P\subseteq S} -M_1 \leq f(P) \leq M_1
\end{gather*}
\end{observation}
\begin{observation} Nie trzeba rozpatrywać wszystkich kombinacji. \\
Dla każdego prefiksu interesuje nas jedynie zbiór uzyskanych różnic częsciowych.
\end{observation}

\end{observation}
\begin{observation} Mając dany zbiór możliwych różnic częsciowych funkcji f dla pierwszych i rodzajów cukierków potrafimy policzyć zbiór możliwych wartości częsciowych dla $i+1$ rodzajów cukierków.
\end{observation}
\begin{observation} Wystarczy pamiętać jedynie dwa ostatnie zbiory.
\end{observation}

\begin{lstlisting}[language=c++]
S[0] = {0}, S[1] = {};
for(i = 1,...,n) {
	S[i mod 2] = {};
	for(v : S[(i+1) mod 2]) {
		for(j = 0,...,k[i]) {
			dc = (2*j - k[i])*c[i];
			S[i mod 2].insert(v + dc);
		}
	}
}
res = INT_MAX;
for(v : S[n%2]) res = min(res,|v|);
\end{lstlisting}
Algorytm działa w pamięci $O(M_1)$ oraz czasie
\begin{description}
	\item $O(KNM_1logM_1)$ z std::set
	\item $O(KNM_1) = O(CK^2N^2)$ z std::unordered\_set 
\end{description}
Dla dużych danych algorytm jest nadal za wolny.

\subsection{Optymalizacja $O(CK^2N)$}
\begin{lemma} Wynik jest ograniczony z góry przez $C$.
\begin{proof}
Pierwszy cukierek dajemy komukolwiek. Kolejne cukierki dajemy zawsze temu który ma mniej. W ten sposób nigdy nie będziemy mieć różnicy większej niż C.
\end{proof}

\end{lemma}
\begin{lemma} Jeśli wartość funkcji na każdym prefiksie jest ograniczona i rośnie, to jeśli wartość na prefiksie przekroczy pewien próg, to wówczas tej kombinacji można nie rozpatrywać.
\begin{proof}
\begin{align*}
&D:= \sum_i k_i c_i \\
&min_{[a]} |\sum_i (2a_i - k_i) c_i| \\
&min_{[a]} |2\sum_i a_i c_i - D| \\
\end{align*}

A więc w interesującym nas przypadku:
\begin{align*}	
& 0 \leq 2\sum_i a_ic_i \leq D + C \leq (K+1)C  \\
& 0 \leq \sum_i a_ic_i \leq \frac{D+C}{2} = M_2 \\
& M_2 = O(CK)
\end{align*}
\end{proof}
\end{lemma}
\begin{displaymath}
\end{displaymath}
Wystarczy więc jedynie zmienić sposób wrzucania do zbioru, oraz nie wrzucać do niego elementów większych od $M_2$. \\
Otrzymaliśmy algorytm $O(KM_2N) = O(CK^2N)$. Niestety, jest on nadal zbyt wolny.
\subsection{Optymalizacja $O(CKN)$}
\begin{observation} Dodając kolejne cukierki tak naprawdę dodajemy osiągalne sumy do zbioru. Można to interpretować jako odwiedzanie, kolejnych sum (wierzchołków).
\end{observation}

W tablicy visited przechowujemy informację o możliwych do osiągnięcia sumach kalorii dla pierwszej osoby. \\
Rozpatrzmy przykładową aktualizację dla cukierka ($M_2 = 16, c=2, k=4$). \\
X oznacza odwiedzenie nowego wierzchołka, \\
O oznacza odwiedzenie wcześniej odwiedzonego wierzchołka \\
$V_i$ to nowe sumy osiągnięte przez dodanie i cukierków danego typu.
\begin{center}
    \begin{tabular}{| l | l | l | l | l | l | p{2cm} |}
    \hline
    $id$ & $visited$ & $V_1(U_1)$ & $V_2(U_0)$ & $V_3(U_1)$ & $V_4(U_0)$ & $visited'$ \\
    \hline
    16 & & & X & & &+ \\
    15 &+& & &O & &+ \\
    14 & &X & & & &+ \\
    13 & & &X & & &+ \\
    12 &+& & & & &+ \\
    11 & &X & & & &+ \\
    10 & & & & & & \\
    9 &+& & &O & &+ \\
    8 & & & & & X &+ \\
    7 & & &X & & &+ \\
    6 & & & &X & &+ \\
    5 & &X & & & &+ \\
    4 & & &X & & &+ \\
    3 &+& & & & &+ \\
    2 & &X & & & &+ \\
    1 & & & & & & \\
    0 &+ & & & & &+ \\
    \hline
    \end{tabular}
\end{center}

\begin{observation} Dwie najbardziej wewnętrzne pętle wykonują zdecydowanie za dużo operacji. Między innymi przepisujemy cały poprzedni zbiór, a następnie odwiedzamy wielokrotnie wcześniej już odwiedzone wierzchołki. Operacja dodania danego typu cukierków kosztowała nas O(KM). Zauważmy, że wystarczy wykonać jedynie O(M) operacji.
\end{observation}

\begin{lstlisting}[language=c++, breaklines=True]
// Odwiedzone
bool visited = (true, false, false, ..., false)
// Kolejki wierzcholkow do zaktualizowania
U[0] = {}, U[1] = {} 
// Dla kazdego typu cukierkow
for (i = 1, ..., n) {
// Oblicz przyrost wynikajacy z dodania jednego nowego cukierka
	dc = c[i];
	for (v = 0,...,M-dc) {
// Dla kazdego nieodwiedzonego wierzcholka, do ktorego da sie dojsc
		if(visited[v] && !visited[v+dc) {
// Oznacz nowy wierzcholek jako odwiedzony
			visited[v+dc] = true;
// Dodaj do kolejki do przetworzenia
			U[1].insert(v);
		}
	}
// Dla kazdego dodanego cukierka
	for(j = 2, ..., k[i]) {
// Stworz nowa pusta kolejke
		U[i%2] = {};
// Dla kazdego wierzcholka z  poprzedniej kolejki
		for(v : U[(i+1)%2]) {
// Znajdz wartosc sumy po dodaniu cukierka
			nv = v + dc; 
// Jesli wczesniej nie uzyskalismy tej sumy.
			if(!visited[nv]) {
 // Oznacz uzyskanie sumy jako odwiedzony.
				visited[nv] = true;
// Dodaj do nowej kolejki
				U[i%2].insert(v);
			}			
		}
	}
}
\end{lstlisting}

Jako kolejkę wystarczy użyc standardowego std::vector.
Powyższy algorytm działa w czasie $O(M_2N) = O(CKN)$.

\end{document}