\documentclass{article}
 
 
\usepackage{polski}
\usepackage[utf8]{inputenc}
\usepackage{amsmath}
\usepackage{amssymb}
\usepackage{listings}
\usepackage{array}
\usepackage{verbatim}
\usepackage{ntheorem}
\lstset{language=C++}

\theoremstyle{break}
\newtheorem{theorem}{Twierdzenie}
\newtheorem{corollary}{Wniosek}[theorem]
\newtheorem{lemma}{Lemat}
\newtheorem{proof}{Dowód}

\usepackage{bbm}
\usepackage{hyperref}
\hypersetup{
    colorlinks=true,
    linkcolor=blue,
    filecolor=magenta,      
    urlcolor=cyan,
}

 
\begin{document}
\title{Kontest}

\maketitle

\section{Zasady}
\begin{enumerate}
\item Można korzystać z dowolnych materiałów.
\item Kod ma być napisany samodzielnie tzn. 'kopiuj-wklej' jest niedozwolone, ale 'przepisz' już jest.
\end{enumerate}
\section{Wskazówki}
\begin{enumerate}
\item Rozwiązania nie powinny zająć więcej niż 60*2 linii (nie wliczając debuga).
\item \href{https://www.oi.edu.pl/old/php/show.php?ac=p173000&module=show&file=oi18/przekierowanie}{Przekierowywanie wejścia, wyjścia z pliku.}
\end{enumerate}

\section{Zadania cz. I}

\subsection{Collatz}
Treść: \url{http://pl.spoj.com/problems/PTCLTZ/} \\
Oczekiwana złożoność: $O(n)$

\subsection{KMP (Knuth Morris Pratt)}
Treść: \url{http://pl.spoj.com/problems/KMP/} \\
Kod na którym można się wzorować: \href{https://gist.github.com/gongzhitaao/5e9d8f80aaba60e14a2c}{Kod pomocniczy} \\
Wskazówki co do rozwiązania: \href{http://students.mimuw.edu.pl/~bw386389/13-str/str.pdf}{Skrypt o KMP} \\
Stopnie rozwiązania zadania:
\begin{enumerate}
\item $O(n^2)$
\item $O(n)$
\end{enumerate}
Kody muszą przechodzić dodatkowe testy postaci:
\begin{enumerate}
\item Wzorzec to prefiksosufiks dłuższego prefiksosufiksu.
\end{enumerate}
\subsubsection{Dodatkowe testy}
\begin{tabular}{l l}
\shortstack[l]{[INPUT] \\1 \\2 \\aa \\aaaaaaaa}&
\shortstack[l]{[OUTPUT]\\0 \\1 \\2 \\3 \\4 \\5 \\6}
\end{tabular}
\\
\subsubsection{Funkcja wypisująca obiekty iterowalne np. vector, string}
\lstinputlisting[language=C++]{pvec.cpp}

\end{document}